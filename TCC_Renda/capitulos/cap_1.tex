% ----------------------------------------------------------

\section{Introdução}
\addcontentsline{toc}{section}{Introdução}
% ----------------------------------------------------------


%\glsxtrfull{IR} \par
%\glsxtrfull{ABNT}\par
%\glsxtrfull{ABNT}\par

%Se a pessoa mexeu com operações na bolsa de valores ela obrigatoriamente necessita fazer a sua declaração de imposto,de acordo com as regras da receita federal.
Como muitos já sabem, investir tem se tornado algo do nosso cotidiano, nos últimos tempos o assunto tem se tornado mais acessível na internet e meios de comunicação e assim muitos jovens veem uma oportunidade de se tornarem independentes financeiramente, olhando para esse ponto pensamos em desenvolver uma ferramenta que facilitaria a vida de muitos investidores tanto amadores quanto os mais profissionais da área.
\par Pensando dessa maneira e no público no qual iremos atender, uma ferramenta foi criada para facilitar a geração dos cálculos que demandam de bastante atenção, tempo e cuidado para serem obtidos, simplificando a declaração dos valores exigidos pelo governo para a declaração do imposto de renda.\par Sabendo das limitações as quais iremos enfrentar durante esse trabalho, iremos utilizar a linguagem Python \cite{borges2014python} e sua biblioteca a qual se chama Numpy \cite{harris2020array}, em que é destinado a fechar a lacuna na riqueza de dados disponíveis, e também utilizamos a biblioteca Pandas \cite{mckinney2015pandas}, que por sua vez é uma ferramenta funcional para computação cientifica, para fazer cálculos de alto nível de diferentes tipos de dados e tem como base nosso principal modelo de dado, o dataframe, visando obter benefício da sua versatilidade e funcionalidades as quais serão de grande auxílio tanto para busca quanto formatação e distinção de dados.
\par Com todas essas ferramentas foram criadas em um sistema que espera do usuário o fornecimento de maneira integra e pré-validada de todas as notas de negociação que ele tem disponível durante sua vida ativa como investidor, pois os cálculos serão construídos através de uma linha do tempo entre todas as negociações que ele realizou na bolsa de valores, assim levantaremos uma carteira que condiz com o seu histórico de movimentações, e que seja capaz de transmitir todo os resultados esperados para a declaração, no fim cabe ao usuário fazer sua conferência para que não haja uma discordância de resultados.

%\section{Objetivos}

%Nas seções abaixo estão descritos o objetivo geral e os objetivos específicos deste TCC.


%\subsection{Objetivo Geral}

%Auxiliar a formatação e resultados para declaração de imposto de renda variável,sobre ações e fundos imobiliários.


%\subsection{Objetivos Específicos}

%Automatizar e formatar utilizando Python e suas funções e biblioteca para geração de valores exatos em sua declaração de imposto de renda.\par
%Demonstrar os ganhos, percas e posição sobre os ativos negociados durante o ano de exercício da declaração de imposto

%\textbf{1 SEÇÃO PRIMÁRIA}

%1.1 SEÇÃO SECUNDÁRIA

%\textbf{1.1.1 Seção terciária}

%\textit{1.1.1.1 Seção quartenária}

%1.1.1.1.1 Seção quinária
%\begin{alineas}
    %\item Seção primária, use o comando \verb|\section{}|.
    %\item Seção secundária, use o comando \verb|\subsection{}|.
    %\item Seção terciária, use o comando \verb|\subsubsection{}|.
    %\item Seção quartenária, use o comando \verb|\subsubsubsection{}|.
    %\item Seção quinária, use o comando \verb|\subsubsubsubsection{}|.
    %\item Título das seções de referências, apêndice e anexo são gerados automaticamente pelo \textit{template}.
    %\item Para citação com mais de três linhas use o comando \verb|\begin{citacao}|.
    %\item Note de rodapé, use o comando \verb|\footnote{}|\footnote{A nota de rodapé é automaticamente formatada pelo \textit{template}.}
%\end{alineas}


