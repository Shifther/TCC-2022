% ----------------------------------------------------------

\section{Metodologia}
% ----------------------------------------------------------

Como investidores iniciantes na área percebemos o ponto fraco encontrado pelos novos investidores, que ainda não possuem conhecimento suficiente e precisam de ajuda para realização de seu Imposto de Renda Pessoa Física (\textbf{IRPF}), visando simplificar a forma de apuração de dados monetários que por sua vez são um grande empecilho para esses investidores, colocamos as principais necessidades encontradas em uma solução de forma prática e intuitiva.
%\glsxtrfull{IRPF 2022} 2022

Utilizando o aplicativo IRPF \cite{IRPF2022}, fornecido pelo Governo Federal e Ministério da Economia, obtivemos as orientações e regras á seguir na hora de fazer a declaração, dentre elas, são apresentados diferentes modelos e requisitos específicos para diferentes condições, baseadas nas movimentações de cada investidor sendo esses aqueles que irão fazer sua declaração de renda, o foco do projeto está em cima da renda variável, enquadrando somente movimentações sobre ações e fundos imobiliários, e sua última constatação de preço médio no ano de exercício.
 
 Com base na necessidade apresentada durante o desenvolvimento do trabalho, teve-se a busca de informação em livros, artigos e muitas informações foram tiradas de manuais de funcionamento, dando enfase na pesquisa por tecnologias com maior fonte de informações disponíveis online.

 Como não encontramos uma base geral para ter acesso aos dados de cada investidor, pensamos em deixar a tarefa de fornecer os dados necessários para o próprio investidor, sendo esses, todas as notas de negociações que ele possui desde que ele começou a investir, contando também a aquisição de ativos através de direitos de subscrição.

Tratando-se da diversidade ligada aos dados, tanto na tipologia como quantidade, escolhemos a linguagem Python \cite{mckinney2019}, por conta de sua flexibilidade com uma gama consideravelmente grande de dados, e também por sua facilidade em análise e termologia de programação.
